\documentclass[12pt]{article}
\usepackage[utf8]{inputenc}
\usepackage{amsmath}
\usepackage{amsfonts}
\usepackage{graphicx}
\usepackage{geometry}
\usepackage{titlesec}
\geometry{a4paper}

\titleformat{\section}[block]{\normalfont\small\bfseries}{\thesection}{1em}{}

\title{AI Debate Systems: Enhancing Structured Argumentation and Interactivity}
\author{Mohamad Faraj Makkawi, Hassan Khan\\
    Supervisors: Mihai Andries, Christophe Lohr\\
    Institution: IMT Atlantique, France}
\date{\today}

\begin{document}

\maketitle

\begin{abstract}
    A debate is a formal discussion where the participants form and support arguments, a good debate must include logical structure, reason from fact, counterarguments, persuasion, and interactivity so AI debate systems must follow structured argumentation models to present clear and logical discussions. Falsified reasoning is also used to reinforce arguments based on more recent evidence, and rebuttal games are interactive. Nature Language Processing based reasoning also enhances the understanding and flexibility of AI arguments. Such approaches enhance the ability of AI systems to engage in relevant and interpretable dialogue.

    Different types of conversational agents that are capable of debating include rule-based agents using pre-defined templates, retrieval-based agents that look up arguments from a database, and hybrid agents that combine structured reasoning with retrieval-based methods. Advanced systems, such as hierarchical persuasion agents, learn how to tailor arguments based on feedback from users, while explainable debate agents construct their reasoning via formal argumentation structures. AI-based debate systems are particularly precious in domains such as healthcare, law, and business, where transparency of reasons is of significant importance. Advanced work must merge hierarchical argumentation, retrieval-based learning, and interactive user engagement to create more persuasive and transparent AI debaters.
\end{abstract}

\section*{Resources for Information}
\begin{small}
    The following resources were used for gathering information: Google Scholar, ResearchGate, IEEE Xplore, and IMT Library.
\end{small}

\section*{References}
\begin{enumerate}
    \item Rakshit, Geetanjali, Kevin K. Bowden, Lena Reed, Amita Misra, and Marilyn Walker. "Debbie, the debate bot of the future." In \textit{Advanced Social Interaction with Agents: 8th International Workshop on Spoken Dialog Systems}, pp. 45-52. Springer International Publishing, 2019.
    \item Tan, Chenhao, Vlad Niculae, Cristian Danescu-Niculescu-Mizil, and Lillian Lee. "Winning arguments: Interaction dynamics and persuasion strategies in good-faith online discussions." In \textit{Proceedings of the 25th International Conference on World Wide Web}, pp. 613-624, 2016.
    \item Kulatska, Iryna. "ArgueBot: Enabling debates through a hybrid retrieval-generation-based chatbot." Master's thesis, University of Twente, 2019.
    \item Lisa A. Chalagune, Anthony Hunter, "A persuasive chatbot using a crowdsourced argument graph and concerns," \textit{Computational Models of Argument 326} (2020): 9.
    \item Sakai, Kazuki, Ryuichiro Higashinaka, Yuichiro Yoshikawa, Hiroshi Ishiguro, and Junji Tomita. "Hierarchical argumentation structure for persuasive argumentative dialogue generation." \textit{IEICE TRANSACTIONS on Information and Systems 103}, no. 2 (2020): 424-434.
    \item Engelmann, Débora, Juliana Damasio, Alison R. Panisson, Viviana Mascardi, and Rafael H. Bordini. "Argumentation as a method for explainable AI: A systematic literature review." In \textit{17th IEEE Iberian Conference on Information Systems and Technologies (CISTI)}, 2022, pp. 1-6.
\end{enumerate}

\end{document}